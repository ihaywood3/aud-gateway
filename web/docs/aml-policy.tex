\documentclass[12pt]{report}
\usepackage{graphicx}
%\usepackage{titlesec} 
\renewcommand{\chaptername}{Part}
\usepackage{cmbright}
\renewcommand{\familydefault}{\rmdefault}
\usepackage[T1]{fontenc}
\begin{document}
\includegraphics{logo} \parbox[b]{5in}{\sffamily RIVER STONE TECHNOLOGIES \\
  64 Robert Dr Mornington 3931\\ w river-stone.com.au \\ e enquiries@river-stone.com }

\rmfamily
\vspace{1.5in}

\begin{center}
  {\Huge \bfseries AML/CTF Programme}
  \vspace{1in}

  {\large Version 1, March 2018}
 \end{center}
\pagenumbering{roman}                   % roman page number for toc
\setcounter{page}{1}                    % make it start with "ii"

 
\renewcommand{\thechapter}{\Alph{chapter}}
\def\labelsection{\thechapter \thesection}

%\renewcommand{\thechapter}{A}%
\chapter{AML/CTF Policy}

\section{Business Structure}

River Stone Technologies is  a small business operating a digital currency exchange.

It is an Australian proprietary company with one director and shareholder,
Dr. Ian Haywood.

The exchange focusses on direct exchange of digital currencies on
the \emph{BitShares} system with Australian currency. It does
not provide overseas services. It does operate its own market.

The company does not have employees and does not operate any physical
branches/outlets, all services are provided online.

Customer interactions are mostly automated by a computer system
(henceforth ``the System'') programmed and operated by Dr. Ian Haywood.

In this document the phrases ``the System will'' and ``the System will not'' mean the
computer system is programmed to automatically do, or to disallow, a particular action.

\section{Purpose of Programme}

The business is a ``digital currency exchange provider'' within the meaning of the
\emph{Anti-Money Laundering and Counter-Terrorism Financing Amendment Act} 2017 (Part 2),
and therefore is required to maintain registration with AUSTRAC, and to have a compliant
AML/CTF programme, which this document describes.

The programme aims to identify with AML/CTF risks the business faces and the steps
taken to reduce and mitigate these risks. The programme identifies the level of risk and
how the programme requirements correspond to the level of risk faced by a facility or the customer group
using it.

\section{Money Laundering and Terror Financing}

Money laundering refers to the practice of criminal organisations to disguise proceeds of crime. A number of techniques
are used to achieve this.

\begin{description}
\item[placement] is introducing funds into the legal financial system;
\item[layering] refers the use of fake companies and other entities, often operating in series, to obscure the origin of funds.
\item[structuring] refers to breaking down a transaction into smaller ones to avoid triggering suspicion or threshold
  safeguards.
\item[integration] refers to using an apparent legitimate transaction to disguise the source of funds.
  \item[identity theft] refers to falsifying the identity of the true owner of funds or operator of the account.
\end{description}

Financial institutions can unwittingly (or, in cases of employee corruption, wittingly) facilitate these transfers
by allowing criminal entities to use their services.

Terror financing is somewhat different in that the funds in question may have a lawful origin (a terrorist
sympathiser may wish to contribute their lawful wages, for example), but have an unlawful \emph{destination}, usually overseas.
The aim is to obscure the destination of the funds and/or the identity of the originator.

The purpose of an AML/CTF programme is to:
\begin{enumerate}
\item make ML/TF activities more difficult to initiate or uninviting;
\item detect possible ML/TF activities as they occur and report;
\item reduce the amount of damage if they do occur (for example transaction size limits);
\item provide records law enforcement can access \emph{post hoc} to trace ML/TF activities.
\end{enumerate}

\section{Specific Risks}

Digital currencies in general are perceived as offering a new level of `privacy' or `secrecy' and have
been used for criminal purposes such as the \emph{Silk Road} case. The specific currency used
by the business is \emph{BitShares}, which offers a relatively lower level of privacy: all transactions
are linked to a named account.

The main anticipated attacks are identity theft, to open accounts under a false name (placement), and
making withdrawals that can be falsely attributed to profits from trading other digital currencies  (integration). 

\section{Countermeasures}

The key element to prevent abuse is to maintain a link between the BitShares account name and
a `real-world' identity. Subsequent transactions (even not with the business) can be linked
and so this reduces the attractiveness of the system to criminal use. This is described in the KYC procedures
in part B.

\subsection{Limitations of the System}

To reduce the overall risk profile, the following limitations apply across the system
(as noted the transaction rules are `enforced' by computer code and don't require employee adherence to
particular rules):
\begin{enumerate}
\item Only Australian residents may open accounts;
\item Accounts may only be opened for individuals and sole traders;
\item Transactions are limited to \$3,000;
\item Each account is limited to \$10,000 turnover per week;
\item No dealings in physical cash;
\item Monies will only be sent to accounts with Australian-based banks;
\item Monies will only be received through the BPay system.
\end{enumerate}

\subsection{Transaction Monitoring}

The company will maintain ongoing surveillance of transactions made with the system. Transactions are automated
and initiated through the \emph{BitShares} system, so there is not in itself any direct customer contact
in each transaction.

However each transaction made through the system is recorded . Each record will have:
\begin{enumerate}
  \item The date and time of the transaction;
  \item The BitShares transaction ID;
  \item The BitShares account name;
  \item the BSB and account number (for withdrawals);
  \item the BPay reference number (for deposits).
\end{enumerate}

All records are maintained for seven years.

Every week, the System will prepare a report of the top 25 accounts (by weekly turnover), and their individual transactions in that week,
for review by the director. The following patterns are considered to be `red flags':
\begin{enumerate}
\item An account suddenly withdrawing or depositing larger amounts compared to its past history;
\item Accounts doing multiple small transfers for no clear reason (i.e. possible structuring);
\item Accounts gaining balance without any correspondence to movements in the digital currency markets;
\end{enumerate}

\subsection{Ongoing Customer Due-Diligence and AUSTRAC Reporting}

Account ``red-flagged'' for their behaviour in this way proceed to ongoing customer due-diligence
\footnote{Some AUSTRAC documentation seems to assume a link between ``ongoing'' and ``enhanced'' coustomer due-diligence.
  The assumption seems to be that an account that behaves susipicously wil lget referred for enhanced KYC procedures. I
  don't view this as appropriate response for the business: suspicious accounts wil ljust get suspended}.
The director will either attempt to contact the customer for an explanation, directly make a Suspicious Matter Report to AUSTRAC,
or make an SMR after contact if the explanation is not satisfactory.

Accounts which give rise to an SMR will be permanently suspended, but records of the transactions and the SMR itself will be retained. Funds
pending will be returned but on the same `side' (digital for digital, fiat for fiat i.e the transaction itself will not be completed).

Any account that tries to break the transfer limits noted above will also be reviewed by the director.

Correspondence will be via e-mail and all e-mails sent to and from customers will be retained for seven years along with other all customer data.

Because the service has a maximum transaction limt well below \$10,000, it is not anticipated the service will need to make Threshold Transaction
Reports.

\subsection{Enhanced Customer Due-Diligence} \label{red-flags}

There is no `enhanced' procedure for high-risk customers who are foreign politically-exposed persons or resident in prescribed countries:
these customers will not be accepted at all.

Enhanced KYC due-diligence applies to customers who undertake `red flag' transactions as applied above, and customers whose initial
application is unusual in some way. `Unusual' is not defined in a restricted way but examples include:
\begin{enumerate}
\item A customer from a demographic not normally associated with digital currency use
  such as an older person, a person who seems to have little computer knowledge,
  limited English speaker (this raises the prospect the true account operator is not the same as the applicant);
\item A customer who refuses or claims to be unable to supply standard identity documents (driver's licence, passport: refer to section
  \ref{standard});
\item a customer who offers poor quality images of documents and claims to be unable to offer better quality;
\end{enumerate}

In the first instance the customer will be invited to explain. If the explanation is complex, evasive or implausible, the customer will be
refused an account. If plausible, they will be invited to complete the `enhanced' or `medium-risk' KYC procedure (see section \ref{enhanced}).
An example of a plausible explanation would be a disabled person unable to drive (so has no driver's licence), who doesn't have a current passport.

Extra KYC information and customer explanations gathered in this way will be recorded on the customers file.

\section{Organisational Matters}

These headings are standard in a AML/CTF programme but do not have much applicability to a small business, but are included for completeness.

\subsection{Roles and Reporting}

As a one-person company the director assumes all responsibility for implementing the programme and tracking issues that arise.

\subsection{Employee Due-Diligence}

The business does not have employees.

The director will submit a National Police Check to AUSTRAC.

\subsection{AML/CTF risk awareness training programme}

The director will undergo training (a one-day TAFE course
or similar) in AML/CTF procedures (specifically KYC procedures) used by customer-facing staff in financial institutions.

The director will be familiar with this Programme and relevant AUSTRAC Rules: this familiarisation has been achieved by writing this document. 

\subsection{AUSTRAC feedback}

Feedback from AUSTRAC will be received by the director (who will be the nominated contact for AUSTRAC). If specific feedback is given this
will precipiate a review of the Programme (see next section) and the feedback will be incorporated into the new programme.

The director will monitor the AUSTRAC website at least six-monthly to see if industry-specific guidance has been released.
The director will also subscribe to AUSTRAC mailing lists, if provided, for this purpose. New guidance 
will precipitate a review if the guidance indicates a problem with the current Programme.

\section{Review of Programme}

This policy will be reviewed if the business structure changes (in particular, if the business takes on any
employees), or if the nature of its products changes,

Failing that, the programme will be reviewed every year.

At review the director will download anew copy of the AUSTRAC Rules and check to see if the requirements have changed.

Within six months of commencing operation, and within six months of each review, the company will
organise an external audit of the programme. The audit response and resulting modifications to the programme will be
recorded.

\chapter{Customer Identification}

\section{Standard (Low-Risk) Procedure} \label{standard}

The standard procedure is for low-risk customers.
\begin{enumerate}
\item All customers complete an application on the company's website. The website will not
  accept applications that are not complete. Information gathered is as follows:
  \begin{enumerate}
  \item the BitShares ID;
  \item e-mail address;
  \item a photograph (usually using a smartphone) of the front of their driver's licence or details page of their passport;
  \item a photograph of the back of their driver's licence (if this was used);
  \item a photograph of a secondary identity document (as defined in the AUSTRAC Rules Chapter 2);
  \item a photograph of the customer themselves (a ``selfie'') holding a piece of paper bearing their BitShares ID;
  \item customers name, date-of-birth and address;
  \end{enumerate}
  \item the application is formatted into an e-mail, encrypted, and sent to the director (all such e-mails are retained);
  \item the director examines the application and checks the following:
    \begin{enumerate}
    \item the images are legible and not modified (``photoshopped''). The resolution of the image must be
      high enough to permit such a determination: at least 300 dpi \footnote{which is about the
        same level of detail the average human eye would get from seeing the original, see
      \texttt{https://jaredjared.com/2012/10/visual-acuity-dpi}};
    \item they must clearly be colour photographs of a complete driver's licence/passport, including the four edges, not a
      flatbed scan or photocopy;
    \item for passports, the secondary identity document must prove the customer's address in Australia;
    \item the photograph of the customer is a reasonable likeness of the photo on the driver's licence/passport;
    \item the applicant is over the age of 18;
    \item the documents match the name and address provided on the application.
    \end{enumerate}
  \item the application is then activated on the system if it matches all the above and does not have `red flags' (see section \ref{red-flags}).
\end{enumerate}

Note only individuals and sole traders can have accounts. There is no KYC procedure for companies, trusts, etc, as such accounts will
not be issued and instead a director/trustee will be invited to apply for an individual account.

\subsection{Rationale for this Method}

The method described above in \ref{standard} is used by most digital currency exchanges both in Australia and overseas, and so follows
industry standard practice.

The AUSTRAC Rules define two `example' methods in sections 4.2.11 and 4.2.13, but the Rules are clear alternative methods are permitted
if they meet risk controls.

\subsubsection{``Documentation-based safe harbour procedure'', 4.2.11}

This method would require that the prospective customer photocopy
identity documents onto paper, have them certified, and then scan the paper copies to submit an application online.

It forms the basis of the `enhanced' KYC procedure (see section \ref{enhanced}), however this method was clearly designed for
applications via post: for an online business it is a considerable extra burden and would not be commercially viable if all customer
applications had to be by this method.

\subsubsection{``Electronic-based safe harbour procedure'', 4.2.13}

This method would require the customer enter the ID numbers from identity documents and these
are verified against a third-party database. This approach was rejected for several reasons:
\begin{enumerate}
\item it does not ascertain that the person operating the remote computer is the same person as the ID holder, i.e. it does not protect well against identity theft;
\item setup costs are considerable for a new business;
\item nowadays it is not significantly faster for a customer (photographing an ID document with a smartphone takes about as long as typing in its serial number);
\item there is a high rate of `false negatives': \emph{bona fide} applications are frequently rejected without a clear reason, which requires a `fallback' method
to  be in place. 
\end{enumerate}

\section{Enhanced (Medium-Risk) Procedure} \label{enhanced}

Customers identified on the initial application as medium-risk are not activated on the system, but instead are
sent a prepared statutory declaration form, in which they declare their name, address and date of birth, and
that they are not operating the account on behalf of another person or using the funds of another person.

The form is signed by the customer and by a witness authorised to witness statutory declarations in their State,
This is returned, along with certified copies of their identity documents. The form will specify that the
witness also declare that the customer is a reasonable likeness to the photographic ID presented, or
that the witness has known the customer (for example, their regular GP). Applications must either
have photograhic ID or such a statement from the witness.

The requirements follow that of section 4.2.11 of the AUSTRAC Rules: namely both a `primary' and a `secondary' identity
document.

`Primary documents' include: birth certificate, naturalisation certificate, passport, proof-of-age card,
and driver's licence. Foreign passports and birth certificates are accepted, but if not in English or French
an accredited translation must be provided.

`Secondary documents' include: utility bills (less than 3 months old), ATO tax statements,
State Government revenue statements (i.e. land tax), or Centrelink statements (all within twelve months). Foreign
secondary identity documents are not accepted (because the whole point is to prove Australian residency).

If the customer declares they don't have any primary identification, and wants to present multiple secondary identification
instead, they will be invited to explain how they opened an Australian-based bank account (which is required to
make any actual use of a any digital currency exchange) and will be expected to provide the same documents on their
application.

As noted in part A, there must be a plausible `story' around why this is the case and the documents provided must
be consistent with the story. The explanation will be recorded on the customer's file.
The documents must prove, at a minimum, the customer's full name and residential address in Australia.

If these requirements cannot be met the application is refused.

\end{document}
